\chapter{Design}
\label{chap:design}

The library allows to solve problems using the branch-and-bound and backtracking methods.
To allow that, the library exports solving functions, as well as several interfaces
(Rust \emph{traits}), which the library user should implement to use solver functions.

This chapter provides a detailed description of the library interface. The full library
documentation can be found in the Appendix.

\section{Interface for basic usage}

\label{sec:basic_usage}

For the basic scenario, the library provides the following types, traits, and functions:

\subsection{Trait \texttt{Subproblem}}

While the library implements common parts of the branch-and-bound and backtracking methods,
it is up to the library user to define the problem to solve. For that, the user must define
their own type to represent a subproblem (a node in the virtual subproblem tree) and implement
the provided \texttt{Subproblem} trait for their type.

The trait \texttt{Subproblem} is defined in the library code as follows:

\begin{lstlisting}[caption=Trait \texttt{Subproblem}]
pub trait Subproblem {
    type Score: Ord;
    fn branch_or_evaluate(&mut self) ->
            SubproblemResolution<Self, Self::Score>;
    fn bound(&self) -> Self::Score;
}
\end{lstlisting}

According to this definition, the library user must provide three things for their type to
implement the \texttt{Subproblem} trait:

\begin{itemize}
 \item An associated type (referred to as \texttt{Score}) that represents the return value of
    the objective and boundary functions. The user-defined \texttt{Score} type must satisfy
    the standard trait \texttt{Ord}, that is, be totally ordered;

 \item The \texttt{branch\_or\_evaluate} method that evaluates the subproblem and returns
    either a sequence of subproblems of the given subproblem, or (if the subproblem can be
    solved directly) the objective function value at the solution to the subproblem. The result
    is returned as a value of the provided polymorphic type \texttt{SubproblemResolution}
    (see the next section for details). \\
    Note: although, logically, this method only queries the state and should not mutate the
    subproblem, a mutable reference to \texttt{self} is provided. This is useful for some
    performance optimizations, e.g., when branching a subproblem: in this case, the library
    is guaranteed to discard the old subproblem object after its subproblems are generated,
    so the library user is allowed to reuse the old object's memory when generating new
    subproblems;

 \item The \texttt{bound} method that evaluates the boundary function at the subproblem
    (gives an upper boundary estimation of the best objective function value achievable for
    the subproblem).
\end{itemize}

\subsection{Enum \texttt{SubproblemResolution}}

\sloppy
For evaluating (resolving) a subproblem, the library provides the \texttt{SubproblemResolution}
polymorphic type. It is defined in the library code as follows:

\begin{lstlisting}[caption=Enum \texttt{SubproblemResolution},label={lst:SubproblemResolution}]
pub enum SubproblemResolution<Node: ?Sized, Score> {
    Branched(Box<dyn Iterator<Item = Node>>),
    Solved(Score),
}
\end{lstlisting}

The enum \texttt{SubproblemResolution} is parametrized with types \texttt{Node}
(which represents a node in the virtual subproblem tree) and \texttt{Score}
(which is the return type of the objective function).
A value of \texttt{SubproblemResolution} is one of the two variants: either \texttt{Branched}
(then it points to an iterator over the sequence of generated subproblems), or \texttt{Solved}
(then it contains the value of the objective function at the solution to the subproblem).

Note: as an extension to the branch-and-bound method as described in chapter \ref{chap:lr},
with \texttt{SubproblemResolution} it is possible to express that a subproblem branches
into an empty sequence. With the regular branch-and-bound method this does not make sense:
every subproblem shall either be split into other (valid) subproblems, or solved directly.
However, rather than only generating subproblems with valid constraints, it may be more
convenient for the library user to generate subproblems, some of which have mutually exclusive
constraints (i.e., have no solution). This is supported by the library:
by branching into an empty sequence, a subproblem signals that no solution exists for it --
in this case, the library merely discards the subproblem. In an implementation strictly
following the method as described in chapter \ref{chap:lr}, such a subproblem would not be
generated in the first place.
% TODO: is ``in the first place'' informal?

\subsection{Enum \texttt{TraverseMethod}}

The \texttt{TraverseMethod} type represents the method of traversing the virtual subproblem
tree. It is defined in the library code as follows:

\begin{lstlisting}[caption=Enum \texttt{TraverseMethod}]
pub enum TraverseMethod<Node> {
    DepthFirst,
    BreadthFirst,
    BestFirst,
    Custom {
        cmp: Box<dyn Fn(&Node, &Node) -> Ordering>,
        cmp_superceeds_bound: bool,
    },
}
\end{lstlisting}

Variants of \texttt{TraverseMethod} represent the following:

\begin{itemize}
 \item \texttt{DepthFirst} -- the virtual subproblem tree is walked in the depth-first search
    manner, and subproblems are processed in the corresponding order (as in backtracking). \\
    Subproblems to be processed are internally stored in a dynamic array (Rust's standard
    collection \texttt{Vec});

 \item \texttt{BreadthFirst} -- the virtual subproblem tree is walked in the breadth-first
    search manner, and subproblems are processed in the corresponding order. \\
    Subproblems to be processed are internally stored in a deque (Rust's standard collection
    \texttt{VecDeque});

 \item \texttt{BestFirst} -- subproblems are processed in the order according to the following
    rule: on every iteration, the most promising subproblem (i.e., the one with highest
    boundary function value) is selected from the set of live nodes and processed. \\
    Subproblems to be processed are internally stored in a binary heap (implemented in the
    external library \texttt{binary\_heap\_plus}). \\
    This method supports early termination: if at some point the selected (most promising)
    subproblem has the boundary value less than the objective function value at the incumbent,
    then, due to properties of the boundary function, this subproblem, as well as all the
    remaining unprocessed subproblems, can be rejected;

 \item \texttt{Custom} -- subproblems are processed in the custom order according to the
    user-specified comparison function \texttt{cmp}: on every iteration, the subproblem that
    is greatest according to \texttt{cmp} is selected from the set of live nodes and
    processed. \\
    Subproblems to be processed are internally stored in a binary heap (implemented in the
    external library \texttt{binary\_heap\_plus}). \\
    \sloppy
    This method supports early termination: the library user may set the
    \texttt{cmp\_superceeds\_bound} flag to tell the library that the order defined by
    the comparison function is stronger than the order defined by the boundary function
    (meaning that ${\forall a, b: a >_{cmp} b \implies a.bound() > b.bound()}$).
    In that case, the same early termination mechanism is applied as for the \texttt{BestFirst}
    method. Otherwise, no early termination is performed.
\end{itemize}

\subsection{Function \texttt{solve}}

The function solves the given problem using the given \texttt{TraverseMethod}.
The function signature in the library code is as follows:

\begin{lstlisting}[caption=Function \texttt{Solve}]
pub fn solve<Node: Subproblem>(
    initial: Node,
    method: TraverseMethod<Node>,
) -> Option<Node>
\end{lstlisting}

The function accepts the initial problem to be solved, which is a polymorphic
parameter of type \texttt{Node}, which is a user-defined type that implements
\texttt{Subproblem}.

The function returns the subproblem that corresponds to the best solution found
(i.e., the one with the highest value of the objective function).

If no subproblem has ever evaluated to an objective function value
(\texttt{SubproblemResolution::Solved}), i.e., all descendant subproblems branched to empty
sequences, the function returns \texttt{None}.

\section{Library extension interface}

\label{sec:lib_extension_interface}

To allow for extension of the library functionality, the following interface is provided:

\subsection{Trait \texttt{BnbAwareContainer}}

The trait represents a container used to store subproblems to be processed
(the set of live nodes). It supports the push operation (to add a subproblem)
and the pop operation (to take and remove a subproblem).
When performing each operation, the container has access to the value of the incumbent's
score (objective function value) at the time of push/pop (if an incumbent exists).

The trait \texttt{BnbAwareContainer} is defined in the library code as follows:

\begin{lstlisting}[caption=Trait \texttt{BnbAwareContainer}]
pub trait BnbAwareContainer<S: Subproblem> {
    fn push_with_incumbent(
        &mut self,
        item: S,
        score: Option<&S::Score>
    );
    fn pop_with_incumbent(
        &mut self,
        score: Option<&S::Score>
    ) -> Option<S>;
}
\end{lstlisting}

A custom \texttt{BnbAwareContainer} can be useful to integrate features such as:

\begin{itemize}
 \item Custom subproblem storage and ordering;
 \item Subproblem filtering based on the score: skipping pushes or pops of rejectable
    subproblems (subproblems when boundary value lower than the incumbent's score). \\
    Note: a container that discards rejectable subproblems on push implements the \emph{eager}
    evaluation strategy; a container that discards rejectable subproblems on pop implements
    the \emph{lazy} evaluation strategy;
 \item Local exit \cite{narkawicz2013formalnasa}: discard subproblems that are too deep in
    the virtual subproblem tree;
 \item Global exit \cite{narkawicz2013formalnasa}: stop the solution attempts after the given
    number of subproblems are processed;
 \item etc.
\end{itemize}

\section{Advanced usage interface}

\label{sec:advanced_usage}

To make use of library extensions (as described in the previous section), the following
additional function is provided:

\subsection{Function \texttt{solve\_with\_container}}

The function signature in the library code is as follows:

\begin{lstlisting}[caption=Function \texttt{solve\_with\_container}]
pub fn solve_with_container<Node, Container>(
    container: Container,
) -> Option<Node>
where
    Node: Subproblem,
    Container: BnbAwareContainer<Node>,
\end{lstlisting}

The function accepts a \texttt{BnbAwareContainer} containing the initial subproblem(s) to
solve. Throughout the solving process, the container is used as the storage for the set of
live nodes.

It returns the subproblem corresponding to the best solution (i.e., the one with the
highest value of the objective function). If no subproblem evaluates to an objective
function value (\texttt{SubproblemResolution::Solved}) before the container refuses to pop
a new subproblem, the function returns \texttt{None}.
