\chapter{Implementation}
\label{chap:impl}

This chapter describes implementation details, such as the reason for using an external library,
additional components that I implemented to test the library, and implementation artifacts.

\section{External dependency}

The library is implemented in pure Rust with an external dependency on the
\texttt{binary\_heap\_plus} library. I decideed to use the external library rahter than the
standard \texttt{BinaryHeap} collection because of of a more convenient API. For example,
the standard \texttt{BinaryHeap} requires that the stored type implements the \texttt{Ord}
trait, while the collection from \texttt{binary\_heap\_plus} can accept comparison functions in
runtime. However, the two implementations of binary heap are conceptually similar, so, with some
extra utility code, the external dependency can be relinquished.

\section{Problem solvers to test the library}

To test the library functionality, I implemented two example solvers that use the library to
solve the boolean satisfaction problem and the knapsack problem. Chapter \ref{chap:intro}
provides the description of the problems and the overview of (library-agnostic)
algorithms to solve them, which my examples are based on.

In addition to that, I implemented solvers for both problems using the same algorithms
but without using the library (\emph{native} solvers).
The native solver for the knapsack problem solves it using the breadth-first search
branch-and-bound method, and the native solver for the boolean satisfaction problem solves it
using the backatracking method.

The implemented solvers can be used to compare performance of library-based and native
implementations, as well as estimate and compare the implementation difficulty for a developer.
