\chapter{Discussion}
\label{chap:discuss}

This chapter suggests applications of the library and discusses its limitations
and possible future work.

\section{Possible applications}

This section summarizes benefits that the library provides and proposes possible applications
of the library.

\subsection{Regular solvers implementation}

The library provides the ability to implement branch-and-bound-based and backtracking-based
problem solvers. So, its most natural application is to use it for implementing such solvers.

As demonstrated in chapter \ref{chap:res}, solver implementations based on the library
tend to have lower code complexity, thus, supposedly, better readability.

In addition to that, as I discovered while testing the example solvers,
the library provides a convenient interface for prototyping, e.g., when one wants to study
what method is most efficient to solve a problem.

When implemented directly,
backtracking-based and branch-and-bound-based algorithms usually have inherently different
code structure: code implementing backtracking is usually written as a recursive function
implicitly relying on the call stack, while branch-and-bound is implemented as a loop
with an explicit FIFO queue or priority queue. For this reason, it would take effort
to rewrite an existing backtracking solution into branch-and-bound or vice versa, especially
if both code bases need to be maintained.

Conversely, with the library (with basic usage, as described in section
\ref{sec:basic_usage}), it takes a single function call argument change to switch between
backtracking, breadth-first branch-and-bound, best-first branch-and-bound, and
branch-and-bound with a custom subproblem processing order.

\subsection{Reliable computing}

Another potential application of the library is to use it to develop problem solvers
that are to be formally verified.

If the library itself is formally verified, i.e., it is proven that problems are solved
correctly, given that a set of properties holds on the input types and their methods,
it can simplify the formal verification of solvers based on the library, as it will be
enough to prove that the required properties hold for the implemented methods.
This is similar to the approch
of authors of \cite{narkawicz2013formalnasa}, \cite{smith2015rigorous}, but the library
implemented in the presented work is more generic and, thus, allows to solve a wider range
of problems.

However, the library genericness is making it harder to verify it. For example,
with advanced usage (as described in sections
\ref{sec:lib_extension_interface} - \ref{sec:advanced_usage}), by supplying a
custom container, one can enable features such as early termination based on
local exit and/or global exit \cite{narkawicz2013formalnasa}. But such modifications to
the algorithm change the meaning of correctness: a different condition is expected to hold
for the solution.

Thus, if the advanced usage of the library is to be verified, perhaps, a different,
more generic approach to the verification itself should be taken.

\subsection{Education}

The use of the library to solve a problem requires the understanding of two areas:
data structures and algorithms, to implement problem-specific logic,
and generic programming, to make use of the generic library interfaces. Thus, the library
could be used in a computer science education program as an exercise on the two concepts.

For example, in Innopolis University, in the second semester of the first year,
Bachelor's students have the Software Systems Analysis and Design course, where
they study design patterns, and the Data Structures and Algorithms course. As the concepts
used by the library are relevant to both courses, a joint lab session or assignment could be
developed, where students would, e.g., solve a problem using the library, or implement and
compare solvers with and without use of the library, etc.

What complicates the integration of the library into the education process, though, is that
it is written in Rust, and only provides a Rust interface, but Rust is not taught at
core courses in Innopolis University, so most students cannot write in it.

\section{Limitations and future work}

This section describes aspects of the library that could be improved.

\subsection{Implementation improvements}

\subsubsection{Performance}

As discussed in chapter \ref{chap:res}, the performance of library-based solvers is lower
than that of native solvers. On the one hand, this is an expected outcome, as through
additional abstract types, the library incurs overhead. On the other hand, thanks to the
Rust compiler optimizations \cite{klabnik2023rust},
%which are capable of optimizing out extra layers of types,
some abstractions are optimized out, resulting in faster execution.

Nevertheless, certain implementation aspects could be improved to achieve better performance.
For example, the library accepts new nodes of a branched subproblem as an object of type
\texttt{Box<dyn Iterator<Item = Node>>}. Because of that, on every branching, the solver
is required to allocate a new \texttt{Box} object.
According to profiling
\footnote{Profiling was performed using \emph{cargo-flamegraph}
(\url{https://github.com/flamegraph-rs/flamegraph})
with the \emph{perf} back-end (\url{https://perfwiki.github.io/main/})},
these allocations are taking about
$0.53\% - 0.97\%$ of the execution time in the Knapsack problem solver,
and could take longer on larger samples.

Instead, at the cost of an additional generic parameter (thus, slightly less convenient
API), static code can be generated for using the iterator returned by a solver, thus,
the \texttt{Box} allocations can be removed completely.

With further profiling and analysis, more possible improvements might be identified.

\subsubsection{Lifetime of user-defined subproblems}

Due to the simplicity of the initial library design, the library implicitly requires
that all user-defined objects that implement \texttt{Subproblem} have the \texttt{'static}
lifetime, meaning that such objects cannot contain references to external
dynamically allocated objects. This limitation occurs because Rust uses subtyping and
variance to manage liftimes, and the \texttt{SubproblemResolution}
(listing \ref{lst:SubproblemResolution}) is defined
such that its \texttt{Node} generic parameter becomes invariant
\footnote{More information on subtyping and variance can be found in the Rust documentation,
electronically available at \url{https://doc.rust-lang.org/nomicon/subtyping.html}}.

Because of that, solver implementations that want to share memory between subproblem objects,
such as the CNF SAT solver implemented as part of this work, have to resort to mechanisms
that move the lifetime management to runtime and inqur additional overhead, e.g.,
use reference counting via \texttt{std::Rc}.

Instead, the library could explicitly allow non-static lifetime for \texttt{Subproblem}
objects in \texttt{SubproblemResolution}, in which case no extra effort would be needed
on the library user side.

\subsection{Design improvements}

\subsubsection{Subproblem generation on branch}

According to the presented library design, a user-defined subproblem, when branching,
must generate and return all of its subproblems at the same time. While it perfectly fits
the branch-and-bound method, it does not alighn well with the backtracking method, where
subproblems are typically generated one by one, on demand. For example, by only ever
considering one subproblem at a time, the native solver of CNF SAT is able to reuse
a single memory block for all the subproblems considered, never allocating or freeing
dynamic memory during the solution process. Conversely, the library-based implementation
of the solver cannot achieve that, as it must create two subproblems from one
at a time, which requires copying.

Instead, the library could use \emph{lending iterators}
(also known as \emph{streaming iterators}) \cite{LendingIterator},
which is a pattern that enables iterators that provide values temporarily,
meaning that the following value can only be taken from a lending iterator after the previous
value is no longer used. Thus, the name: an iterator \emph{lends} a value,
but the user of the iterator must ``give it back'' (i.e., give up access to it)
before they can access the following one.

With lending iterators, it is possible for a library user to implement a solver that
reuses memory: having the guarantee that the following element is only accessed when
the previous one is no longer accessible, every subproblem may rely on the same memory
region. The region can only be accessed externally through the currently lended subproblem.
Yet, with lending iterators, the library can still implement the branch-and-bound method
by copying lended subproblems according to a user-specified algorithm, via the standard
\texttt{Clone} trait.

Thus, subproblem generation and copying are separated, and, although both still need to be
performed with the branch-and-bound method, allocation and copying can be avoided with
backtracking.

However, lending iterators are not part of standard Rust (yet?), and are only available as
community crates
\footnote{\url{https://docs.rs/lending-iterator/latest/lending_iterator/}}
\footnote{\url{https://docs.rs/streaming-iterator/latest/streaming_iterator/}},
so their use in the library interface might complicate the API.
